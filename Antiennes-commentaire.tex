\documentclass[%
fontsize=10%
,a6paper%
,DIV=15%
]{scrartcl}
%scrartcl



\usepackage{gredocument}
\usepackage{psaume}

\title{\centrer{Grandes Antiennes de l'Avent}}
\author{ou Antiennes O}
\date{du 17 au 23 décembre}

\makeindex
\definecolor{rubrum}{rgb}{.6,0,0}
\def\rubrum{\color{black}}%%%%%%%mettre"\def\rubrum{\color{rubrum}}" pour avoir le texte adéquat en rouge
\def\nigra{\color{black}}
%    \redlines %%%avtiver ce paramètre pour rendre rouge les lignes de partition
%    \definecolor{gregoriocolor}{rgb}{.6,0,0}
%
%\let\red\rubrum
    \newfontfamily\lettrines[Scale=1.3]{LettrinesPro800}
    \def\gretextformat#1{{\fontsize{\taillepolice}{\taillepolice}\selectfont #1}}
    \def\greinitialformat#1{{\lettrines #1}}
%%%%%%%%%%%%%%%%%%%%%%%%%%%%%%%%%%%%%%%%%%%%%%%%%%%%%%%%%%%%%%%%%%%%%%%%%%%%%%%%%%%%%%%%%%%%%%%%%
\begin{document}

\newcommand{\ligne}[2]{
\begin{center}
\greseparator{#1}{#2}
\end{center}
}
\thispagestyle{empty}
\maketitle
\thispagestyle{empty}
%%%%%%%%%%%%%%%%%%%%%%%%%%%%%%%%%%%
\titre{Introduction}

L’Église ouvre aujourd’hui la série septénaire des jours qui précèdent la Vigile de Noël, et qui sont célèbres dans la Liturgie sous le nom de Féries majeures. L’Office ordinaire de l’Avent prend plus de solennité ; les Antiennes des Psaumes, à Laudes et aux Heures du jour, sont propres au temps et ont un rapport direct avec le grand Avènement. Tous les jours, à Vêpres, on chante une Antienne solennelle qui est un cri vers le Messie, et dans laquelle on lui donne chaque jour quelqu’un des titres qui lui sont attribués dans l’Écriture.

Le nombre de ces Antiennes, qu’on appelle vulgairement les \emph{O} de l’Avent, parce qu’elles commencent toutes par cette exclamation, est de sept dans l’Église romaine, une pour chacune des sept Féries majeures, et elles s’adressent toutes à Jésus-Christ. D’autres Églises, au Moyen-Age, en ajoutèrent deux autres : une à la Sainte Vierge, \emph{O Virgo Virginum !} et une à l’Ange Gabriel, \emph{O Gabriel !} ou encore à saint Thomas, dont la fête tombe dans le cours des Féries majeures. Cette dernière commence ainsi : \emph{O Thomas Didyme !}

Il y eut même des Églises qui portèrent jusqu’à douze le nombre des grandes Antiennes, en ajoutant aux neuf dont nous venons de parler, trois autres, savoir : une au Christ, \emph{O Rex pacifice !} une seconde à la Sainte Vierge, \emph{O mundi Domina !} et enfin une dernière en manière d’apostrophe à Jérusalem, \emph{O Hierusalem !}

L’instant choisi pour faire entendre ce sublime appel à la charité du Fils de Dieu, est l’heure des Vêpres, parce que c’est sur le Soir du monde, \emph{vergente mundi vespere}, que le Messie est venu. On les chante à Magnificat, pour marquer que le Sauveur que nous attendons nous viendra par Marie. On les chante deux fois, avant et après le Cantique, comme dans les fêtes Doubles, en signe de plus grande solennité ; et même l’usage antique de plusieurs Églises était de les chanter trois fois, savoir : avant le Cantique lui-même, avant \emph{Gloria Patri}, et après \emph{Sicut erat}. Enfin, ces admirables Antiennes, qui contiennent toute la moelle de la Liturgie de l’Avent, sont ornées d’un chant plein de gravité et de mélodie ; et les diverses Églises ont retenu l’usage de les accompagner d’une pompe toute particulière, dont les démonstrations toujours expressives varient suivant les lieux. Entrons dans l’esprit de l’Église et recueillons-nous, afin de nous unir, dans toute la plénitude de notre cœur, à la sainte Église, lorsqu’elle fait entendre à son Époux ces dernières et tendres invitations auxquelles il se rend enfin.

\titre{17 décembre}
\vulgo{O Sagesse, sortie de la bouche du Très-Haut, qui enveloppez toutes choses d'un pôle à l'autre et les disposez avec force et douceur, venez nous enseigner le chemin de la prudence.}
O Sagesse incréée qui bientôt allez vous rendre visible au monde, qu’il apparaît bien en ce moment que vous disposez toutes choses ! Voici que, par votre divine permission, vient d’émaner un Édit de l’empereur Auguste pour opérer le dénombrement de l’univers. Chacun des citoyens de l’Empire doit se faire enregistrer dans sa ville d’origine. Le prince croit dans son orgueil avoir ébranlé à son profit l’espèce humaine tout entière. Les hommes s’agitent par millions sur le globe, et traversent en tous sens l’immense monde romain ; ils pensent obéir à un homme, et c’est à Dieu qu’ils obéissent. Toute cette grande agitation n’a qu’un but : c’est d’amener à Bethléhem un homme et une femme qui ont leur humble demeure dans Nazareth de Galilée ; afin que cette femme inconnue des hommes et chérie du ciel, étant arrivée au terme du neuvième mois depuis la conception de son fils, enfante à Bethléhem ce fils dont le Prophète a dit : « Sa sortie est dès les jours de l’éternité ; ô Bethléhem ! tu n’es pas la moindre entre les mille cités de Jacob ; car il sortira aussi de toi. » O Sagesse divine ! Que vous êtes forte, pour arriver ainsi à vos fins d’une manière invincible quoique cachée aux hommes ! Que vous êtes douce, pour ne faire néanmoins aucune violence à leur liberté ! Mais aussi, que vous êtes paternelle dans votre prévoyance pour nos besoins ! Vous choisissez Bethléhem pour y naître, parce que Bethléhem signifie la Maison du Pain. Vous nous montrez par là que vous voulez être notre Pain, notre nourriture, notre aliment de vie. Nourris d’un Dieu, nous ne mourrons plus désormais. O Sagesse du Père, Pain vivant descendu du ciel, venez bientôt en nous, afin que nous approchions de vous, et que nous soyons illuminés de votre éclat ; et donnez-nous cette prudence qui conduit au salut.

\cantus{Antienne}{OSapientia}{}{2.D.}
\rubrica{Magnificat page \pageref{section:magnificat}}
\ligne{2}{10}

\titre{18 décembre}
\vulgo{O Adonaï, guide du peuple d'Israël, qui êtes apparu à Moïse dans le feu du buisson ardent, et lui avez donné vos commandements sur le mont Sinaï, armez votre bras, et venez nous sauver.}
O Seigneur suprême ! Adonaï ! Venez nous racheter, non plus dans votre puissance, mais dans votre humilité. Autrefois vous vous manifestâtes à Moïse, votre serviteur, au milieu d’une flamme divine ; vous donnâtes la Loi à votre peuple du sein des foudres et des éclairs : maintenant il ne s’agit plus d’effrayer, mais de sauver. C’est pourquoi votre très pure Mère Marie ayant connu, ainsi que son époux Joseph, l’Édit de l’Empereur qui va les obliger d’entreprendre le voyage de Bethléhem, s’occupe des préparatifs de votre heureuse naissance. Elle apprête pour vous, divin Soleil, les humbles langes qui couvriront votre nudité, et vous garantiront de la froidure dans ce monde que vous avez fait, à l’heure où vous paraîtrez, au sein de la nuit et du silence. C’est ainsi que vous nous délivrerez de la servitude de notre orgueil, et que votre bras se fera sentir plus puissant, alors qu’il semblera plus faible et plus immobile aux yeux des hommes. Tout est prêt, ô Jésus ! Vos langes vous attendent : partez donc bientôt et venez en Bethléhem nous racheter des mains de notre ennemi.

\cantus{Antienne}{OAdonai}{}{2.D.}
\rubrica{Magnificat page \pageref{section:magnificat}}
\ligne{2}{10}

\titre{19 décembre}
\vulgo{O Fils de la race de Jessé, signe dressé devant les peuples, vous devant qui les souverains resteront silencieux, vous que les peuples appelleront au secours, délivrez-nous, venez, ne tardez plus !}
Vous voici donc en marche, ô Fils de Jessé, vers la ville de vos aïeux. L’Arche du Seigneur s’est levée et s’avance, avec le Seigneur qui est en elle, vers le lieu de son repos. « Qu’ils sont beaux vos pas, ô Fille du Roi, dans l’éclat de votre chaussure », lorsque vous venez apporter leur salut aux villes de Juda ! Les Anges vous escortent, votre fidèle Époux vous environne de toute sa tendresse, le ciel se complaît en vous, et la terre tressaille sous l’heureux poids de son Créateur et de son auguste Reine. Avancez, ô Mère de Dieu et des hommes, Propitiatoire tout-puissant où est contenue la divine Manne qui garde l’homme de la mort ! Nos cœurs vous suivent, vous accompagnent, et, comme votre Royal ancêtre, nous jurons « de ne point entrer dans notre maison, de ne point monter sur notre couche, de ne point clore nos paupières, de ne point donner le repos à nos tempes, jusqu’à ce que nous ayons trouvé dans nos cœurs une demeure pour le Seigneur que vous portez, une tente pour le Dieu de Jacob ». Venez donc, ainsi voilé sous les flancs très purs de l’Arche sacrée, ô rejeton de Jessé, jusqu’à ce que vous en sortiez pour briller aux yeux des peuples, comme un étendard de victoire. Alors les rois vaincus se tairont devant vous, et les nations vous adresseront leurs vœux. Hâtez-vous, ô Messie ! Venez vaincre tous nos ennemis, et délivrez-nous.

\cantus{Antienne}{ORadixIesse}{}{2.D.}
\rubrica{Magnificat page \pageref{section:magnificat}}
\ligne{2}{10}

\titre{20 décembre}
\vulgo{O Clef de la cité de David, sceptre du royaume d'Israël, vous ouvrez, et personne alors ne peut fermer ; vous fermez, et personne ne peut ouvrir ; venez, faites sortir du cachot le prisonnier établi dans les ténèbres et la nuit de la mort.}
O Fils de David, héritier de son trône et de sa puissance, vous parcourez, dans votre marche triomphale, une terre soumise autrefois à votre aïeul, aujourd’hui asservie par les Gentils. Vous reconnaissez de toutes parts, sur la route, tant de lieux témoins des merveilles de la justice et de la miséricorde de Jéhovah votre Père envers son peuple, au temps de cette ancienne Alliance qui tire à sa fin. Bientôt, le nuage virginal qui vous couvre étant ôté, vous entreprendrez de nouveaux voyages sur cette même terre ; vous y passerez en faisant le bien, et guérissant toute langueur et toute infirmité, et cependant n’ayant pas où reposer votre tête. Du moins, aujourd’hui, le sein maternel vous offre encore un asile doux et tranquille, où vous ne recevez que les témoignages de l’amour le plus tendre et le plus respectueux. Mais, ô Seigneur ! il vous faut sortir de cette heureuse retraite ; il vous faut, Lumière éternelle, luire au milieu des ténèbres ; car le captif que vous êtes venu délivrer languit dans sa prison. Il s’est assis dans l’ombre de la mort, et il y va périr, si vous ne venez promptement en ouvrir les portes avec votre Clef toute puissante ! Ce captif, ô Jésus, c’est le genre humain, esclave de ses erreurs et de ses vices : venez briser le joug qui l’accable et le dégrade ; ce captif, c’est notre cœur trop souvent asservi à des penchants qu’il désavoue : venez, ô divin Libérateur, affranchir tout ce que vous avez daigné faire libre par votre grâce, et relever en nous la dignité de vos frères.

\cantus{Antienne}{OClavisDavid}{}{2.D.}
\rubrica{Magnificat page \pageref{section:magnificat}}
\ligne{2}{10}

\titre{21 décembre}
\vulgo{O Orient, splendeur de la Lumière éternelle, Soleil de justice, venez, illuminez ceux qui sont assis dans les ténèbres et la nuit de la mort.}
Divin Soleil, ô Jésus ! Vous venez nous arracher à la nuit éternelle : soyez à jamais béni ! Mais combien vous exercez notre foi, avant de luire à nos yeux dans toute votre splendeur ! Combien vous aimez à voiler vos rayons, jusqu’à l’instant marqué par votre Père céleste, où vous devez épanouir tous vos feux ! Voici que vous traversez la Judée ; vous approchez de Jérusalem ; le voyage de Marie et de Joseph tire à son terme. Sur le chemin, vous rencontrez une multitude d’hommes qui marchent en toutes les directions, et qui se rendent chacun dans sa ville d’origine, pour satisfaire à l’Édit du dénombrement. De tous ces hommes, aucun ne vous a soupçonné si près de lui, ô divin Orient ! Marie, votre Mère, est estimée par eux une femme vulgaire ; tout au plus, s’ils remarquent la majesté et l’incomparable modestie de cette auguste Reine, sentiront-ils vaguement le contraste frappant entre une si souveraine dignité et une condition si humble ; encore ont-ils bientôt oublié cette heureuse rencontre. S’ils voient avec tant d’indifférence la mère, le fils non encore enfanté à la lumière visible, lui donneront-ils une pensée ? Et cependant ce fils, c’est vous-même, ô Soleil de justice ! Augmentez en nous la Foi, mais accroissez aussi l’amour. Si ces hommes vous aimaient, ô libérateur du genre humain, vous vous feriez sentir à eux ; leurs yeux ne vous verraient pas encore, mais du moins leur cœur serait ardent dans leur poitrine, ils vous désireraient, et ils hâteraient votre arrivée par leurs vœux et leurs soupirs. O Jésus qui traversez ainsi ce monde que vous avez fait, et qui ne forcez point l’hommage de vos créatures, nous voulons vous accompagner dans le reste de votre voyage ; nous baisons sur la terre les traces bénies des pas de celle qui vous porte en son sein ; nous ne voulons point vous quitter jusqu’à ce que nous soyons arrivés avec vous à l’heureuse Bethléhem, à cette Maison du Pain, où enfin nos yeux vous verront, ô Splendeur éternelle, notre Seigneur et notre Dieu !

\cantus{Antienne}{OOriens}{}{2.D.}
\rubrica{Magnificat page \pageref{section:magnificat}}
\ligne{2}{10}

\titre{22 décembre}
\vulgo{O Roi des nations, objet de leur désir, clef de voûte qui unissez les peuples opposés, venez sauver l'homme que vous avez façonné d'argile.}
O Roi des nations ! Vous approchez toujours plus de cette Bethléhem où vous devez naître. Le voyage tire à son terme, et votre auguste Mère, qu’un si doux fardeau console et fortifie, va sans cesse conversant avec vous par le chemin. Elle adore votre divine majesté, elle remercie votre miséricorde ; elle se réjouit d’avoir été choisie pour le sublime ministère de servir de Mère à un Dieu. Elle désire et elle appréhende tout à la fois le moment où enfin ses yeux vous contempleront. Comment pourra-t-elle vous rendre les services dignes de votre souveraine grandeur, elle qui s’estime la dernière des créatures ? Comment osera-t-elle vous élever dans ses bras, vous presser contre son cœur, vous allaiter à son sein mortel ? Et pourtant, quand elle vient à songer que l’heure approche où, sans cesser d’être son fils, vous sortirez d’elle et réclamerez tous les soins de sa tendresse, son cœur défaille et l’amour maternel se confondant avec l’amour qu’elle a pour son Dieu, elle est au moment d’expirer dans cette lutte trop inégale de la faible nature humaine contre les plus fortes et les plus puissantes de toutes les affections réunies dans un même cœur. Mais vous la soutenez, ô Désiré des nations ! Car vous voulez qu’elle arrive à ce terme bienheureux qui doit donner à la terre son Sauveur, et aux hommes la Pierre angulaire qui les réunira dans une seule famille. Soyez béni dans les merveilles de votre puissance et de votre bonté, ô divin Roi ! et venez bientôt nous sauver, vous souvenant que l’homme vous est cher, puisque vous l’avez pétri de vos mains. Oh ! Venez, car votre œuvre est dégénérée ; elle est tombée dans la perdition ; la mort l’a envahie : reprenez-la dans vos mains puissantes, refaites-la ; sauvez-la ; car vous l’aimez toujours, et vous ne rougissez pas de votre ouvrage.

\cantus{Antienne}{ORexGentium}{}{2.D.}
\rubrica{Magnificat page \pageref{section:magnificat}}
\ligne{2}{10}

\titre{23 décembre}
\vulgo{O Emmanuel, notre roi et législateur, que tous les peuples attendent comme leur Sauveur, venez nous sauver, Seigneur notre Dieu !}
O Emmanuel ! Roi de Paix ! Vous entrez aujourd’hui dans Jérusalem, la ville de votre choix ; car c’est là que vous avez votre Temple. Bientôt vous y aurez votre Croix et votre Sépulcre ; et le jour viendra où vous établirez auprès d’elle votre redoutable tribunal. Maintenant, vous pénétrez sans bruit et sans éclat dans cette ville de David et de Salomon. Elle n’est que le lieu de votre passage, pour vous rendre à Bethléhem. Toutefois, Marie votre mère, et Joseph son époux, ne la traversent pas sans monter au Temple, pour y rendre au Seigneur leurs vœux et leurs hommages : et alors s’accomplit, pour la première fois, l’oracle du prophète Aggée qui avait annoncé que la gloire du second Temple serait plus grande que celle du premier. Ce Temple, en effet, se trouve en ce moment posséder une Arche d’Alliance bien autrement précieuse que celle de Moïse, mais surtout incomparable à tout autre sanctuaire qu’au ciel même, par la dignité de Celui qu’elle contient. C’est le Législateur lui-même qui est ici, et non plus simplement la table de pierre sur laquelle la Loi est gravée. Mais bientôt l’Arche vivante du Seigneur descend les degrés du Temple, et se dispose à partir pour Bethléhem, où l’appellent d’autres oracles. Nous adorons, ô Emmanuel, tous vos pas à travers ce monde, et nous admirons avec quelle fidélité vous observez ce qui a été écrit de vous, afin que rien ne manque aux caractères dont vous devez être doué, ô Messie, pour être reconnu par votre peuple. Mais souvenez-vous que l’heure est près de sonner, que toutes choses se préparent pour votre Nativité, et venez nous sauver ; venez, afin d’être appelé non plus seulement Emmanuel, mais Jésus, c’est-à-dire Sauveur.

\cantus{Antienne}{OEmmanuel}{}{2.D.}
\rubrica{Magnificat page \pageref{section:magnificat}}
\ligne{2}{10}

\newpage
\titre{Magnificat}
\label{section:magnificat}
\cantus{Psaume}{Intonation-Magnificat-2D}{2.D.}{}
\canticum[tonus=per,primus=2,ultimus=0,numerus=\value{numerus}]{Magnificat}
\gloria[tonus=per]
\medskip
\rubrica{On reprend alors l'antienne, puis le chantre dit le verset :}
\medskip
\versio{\V Roráte cæli désuper et nubes pluant justum.}{\V Cieux répandez d'en haut votre rosée, et que les nuées fassent pleuvoir le Juste}
\versio{\textbf{\R Aperiátur terra et gérminet Salvatórem.}}{\textbf{\R Que la terre s'entrouvre et fasse germer le Sauveur}}
\versio{Orémus}{Prions}

\rubrica{avant le 4\ieme{} dimanche de l'Avent (en dehors des Quatre-Temps)}
\versio{Aurem tuam, quǽsumus,
Dómine, précibus nostris
accómmoda : et mentis nostræ
ténebras, grátia tuæ visitatiónis
illústra: Qui vivis.}
{Seigneur, prêtez l’oreille à nos prières : et quand vous nous ferez la grâce de venir parmi nous, apportez votre lumière dans l’obscurité de nos âmes.}

\rubrica{le mercredi des Quatre-Temps :}
\versio{Pr\'æsta, quaésumus, omnípotens
Deus : ut redemptiónis
nostræ ventúra sollémnitas et
præséntis nobis vitæ subsídia
cónferat, et ætérnæ beatitúdinis
praémia largiátur. Per Dóminum}{Faites, nous vous le demandons, Seigneur : que la solennité approchant de notre rédemption nous ap\-por\-te avec elle les grâces nécessaires pour la vie présente et nous obtienne la récompense du bonheur éternel.}

\rubrica{le vendredi des Quatre-Temps :}
\versio{Excita, quǽsumus, Dómine, poténtiam tuam, et veni : ut hi, qui in tua pietáte confídunt, ab omni cítius adversitáte liberéntur : Qui vivis.}{Excitez votre puissance, Seigneur, et venez, pour que vos fidèles confiants en votre bonté, soient très vite délivrés de tout ce qui leur fait obstacle.}


\rubrica{à partir du samedi avant le 4\ieme{} dimanche de l'Avent}
\versio{Excita, quǽsumus, Dómine, poténtiam tuam, et veni : et magna nobis virtúte succúrre ; ut per auxílium grátiæ tuæ, quod nostra peccáta præpédiunt, indulgéntiæ tuæ propitiatiónis accéleret : Qui vivis et regnas.}
{Excitez, Seigneur, votre puissance et venez : donnez-nous le secours de votre force infinie, et qu’avec l’aide de votre grâce, votre indulgente bonté nous accorde sans délai ce que retardent nos péchés.}
%\begin{center}
%\subsection*{\centering{Test de titre}}
%\end{center}
%Et maintenant le texte
%\titre[espace=5,table=LaTable,matiere=idem]{Et le deuxième titre}
%Et le deuxième texte
%\addcontentsline{toc}{chapter}{chapitre}
%\addcontentsline{toc}{part}{section1}
%\addcontentsline{toc}{section}{La Section}

%\addcontentsline{toc}{subsection}{subsection1}
%\addcontentsline{toc}{subsubsection}{sous-soussection}

%\tableofcontents
\end{document}