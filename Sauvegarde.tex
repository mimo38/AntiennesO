\documentclass[%
fontsize=12%
,a5paper%
,DIV=15%
]{scrartcl}

\usepackage{gredocument}
\usepackage{mudoc}
\usepackage{adaptateur}

\title{Rituel à l'usage des fidèles}
\author{}
\date{}

%\title{\textsc{\Huge{Rituel à l'usage des fidèles}}}
%\date{}

%\addcontentsline{toc}{niveau}{titre}


\begin{document}
\maketitle
\newpage
\chap
\titre{Objet de ce fascicule}
La Sainte Église, Notre Mère,a renfermé dans six livres principaux
l'ensemble des actes religieux qui se rapportent au
culte de Dieu et qu'on appelle pour cette raison les \emph{Livres
liturgiques}. Ce sont le Missel, le Bréviaire, le Rituel, le Pontifical,
le Cérémonial des Évêques et le Martyrologe.

Le \emph{Rituel} renferme les rites sacrés qui accompagnent l'administration
des Sacrements et des autres fonctions saintes accomplies
pour attirer sur nous les bénédictions du ciel. De là
le nom de \emph{Rituel des Fidèles}, donné à la présente publication,
destinée à instruire plus parfaitement les fidèles sur les Rites
de ces différents actes liturgiques.

\titre{Avantages}
L'Église veut que les fidèles soient instruits parfaitement
de ces rites sacrés. Le Concile de Trente, en effet, s'exprime
ainsi sur ce point  :

\og Afin,que le peuple s'approche des Sacrements avec plus
de respect et de ferveur, le saint Concile ordonne à tous les
évêques qui auraient à les administrer d'en expliquer auparavant,
et de manière à être compris, la pratique et l 'efficacité.
Ils veilleront aussi à ce que les curés, si cela se peut commodément
et s'il en est besoin, donnent les mêmes explications avec
beaucoup de sagesse et de piété et en langue vulgaire.\fg \footnote{Concile de Trente, Sess. 24, De Rej., chap. vii}

Pour faciliter cet enseignement, rendu souvent difficile
aujourd'hui par les nécessités plus urgentes d'une prédication
tout à fait élémentaire, nous avons voulu vulgariser la liturgie
des Sacrements. Ces explications sur l'efficacité, la pratique,
l 'utilité et les rites des Sacrements ne peuvent qu'instruire,
édifier et bien préparer les fidèles à les recevoir. Il en est tant
qui, plus ou moins instruits de la doctrine sacramentelle, ne
comprennent rien aux cérémonies dont ils sont les témoins.
\og Ces rites, cependant, dit le Catéchisme du Concile de Trente\footnote{2e partie, n$^o$ 16}, expriment les effets des Sacrements et les
rendent comme sensibles aux yeux des fidèles qui en comprennent mieux la sainteté. Leur foi, leur charité, leurs sentiments
surnaturels en sont excités davantage; aussi faut-il
avoir soin que ces cérémonies si touchantes et si instructives
ne leur soient pas inconnues.\fg

\begin{center}
\chap
\titreb{SACREMENTS}
\end{center}
\titrec{Communion des infirmes}
\rubrique{En entrant dans la chambre du malade, le prêtre dit :}
\pax
\rubrique{Puis après avoir déposé le Saint-Sacrement à un endroit que l'on aura pu préparer en y mettant une nappe blanche, un crucifix et deux cierges, il asperge le malade, la chambre et les assistants qui se mettent à genoux s'ils le peuvent}
\asperges
\rubrique{Puis après l'éventuelle confession du malade, on récite le confiteor s'il n'a pas déjà été récité}
\label{confiteor}
\includescore{sources/confiteor}
\traduire{\V Misereátur tui omnípotens Deus, et, dimíssis
peccátis tuis, perdúcat te ad vitam ætérnam.}{\V Que le Dieu Tout-Puissant ait
pitié de vous, qu'il vous
pardonne vos péchés et vous
conduise à la vie éternelle.}
\Amen
\traduire{Indulgéntiam,~{\x}~absolutiónem, et remissiónem
peccatórum tu\'orum tríbuat tibi omnípotens
et miséricors Dóminus.}{Que le Seigneur Tout-Puissant et miséricordieux vous
accorde le pardon,~{\x}~l'absolution
et la rémission de vos péchés.}
\Amen
\rubrique{Puis, tenant en main la sainte Hostie et la montrant, le prêtre dit :}
\traduire{Ecce Agnus Dei, ecce qui tollit pecc\'ata mundi}{Voici l'Agneau de Dieu, voici Celui qui efface les péchés du monde.}
\rubrique{On dit alors trois fois}
\traduire{\textbf{\R Dómine, non sum dignus,
ut intres sub tectum meum:
sed tantum dic verbo,
et sanábitur ánima mea}}{\textbf{\R Seigneur, je ne suis pas
digne que vous entriez sous
mon toit, mais dites seulement
une parole et mon âme sera
guérie.}}

\traduire{Corpus D\'omini nostri Jesu
Christi cust\'odiat \'animam
tuam in vitam ætérnam. Amen.}{Que le corps de Notre-Seigneur Jésus-Christ garde votre âme pour la vie éternelle. Ainsi soit-il.}

\rubrique{Si la communion est donnée sous forme de viatique, le prêtre utilise alors la formule suivante :}
\traduire{Accipe, frater (\emph{vel soror}), Vi\'aticum Corporis
D\'omini nostri Jesu Christi,
qui te custodiat ab hoste
mal\'igno, et perd\'ucat in vitam ætérnam.
Amen.}
{Recevez, mon frère \emph{(ou
ma soeur)}, le viatique du
Corps de notre Seigneur Jésus-Christ; qu'il vous défende contre l'ennemi, et
vous conduise à la vie éternelle. Ainsi soit-il.}

\medskip
\traduire{\dominus}{\leseigneur}
\traduire{Orémus.}{Prions.}
\traduire{Dómine sancte, Pater omnípotens, ætérne Deus, te fidéliter deprecámur, ut accipiénti fratri nostro \emph{(soróri nostræ)} sacrosánctum Corpus Dómini nostri Jesu Christi Fílii tui, tam córpori,
quam ánimæ prosit ad remédium sempitérnum: Qui tecum vivit et regnat in unitáte Spíritus
Sancti Deus, per ómnia s\'æcula sæculórum.}
{Seigneur saint, Père tout puissant,
Dieu éternel, nous vous demandons avec
confiance que le Corps sacré
de notre Seigneur JésusChrist,
votre Fils, que notre
frère \emph{(ou soeur)} vient de
recevoir lui soit, tant pour
le corps que pour l'âme, un
remède perpétuel. Lui qui vit
et règne avec vous en l'unité
du Saint-Esprit, Dieu dans
tous les siècles des siècles.}
\Amen

\rubrique{Avant de se retirer, le prêtre bénit en silence le malade avec
le ciboire ou la pyxide, s'il y reste quelque parcelle consacrée;
sinon, il le bénit de la main, disant:}
\traduire{Benedíctio Dei omnipoténtis, Patris, {\x}  et Fílii, et Spíritus Sancti,
descéndat super te et m\'aneat semper.}
{Que la bénédiction du Dieu tout-puissant : du Père {\x}  et du Fils, et du Saint-Esprit, descende sur vous
et demeure toujours en vous.}
\Amen

%\titrec{Rituel continu d'administration du sacrement des malades}
\titrec{Extrême-Onction}
\textit{\small Le sacrement de l'extrême-onction a été institué par notre
Seigneur Jésus-Christ pour le soulagement spirituel et
corporel des malades. Les onctions faites sur les sens du
malade avec l'huile des infirmes, spécialement consacrée
à cet effet, signifient que le sacrement guérit les blessures
de l'âme tout comme l'huile guérit les plaies du corps. Plus
spécialement envisagés, les effets de l'extrême-onction sont
de purifier de toutes ses fautes le chrétien mourant, de le
fortifier contre les tentations de la dernière heure et de
l'aider à mourir saintement. En même temps, ce sacrement
adoucit les souffrances du malade, et peut même lui rendre
la santé si Dieu le juge utile au salut de son âme.
Aussi ne faut-il retarder sous aucun prétexte l'administration
de ce sacrement aux malades en danger de mort.
La crainte de les impresssionner n'a que le triste résultat
de les priver de grâces précieuses et de compromettre leur
salut éternel.
L'extrême-onction peut et doit être réitérée chaque fois
que, après une amélioration de son état, le malade retombe
dans un nouveau danger de mort.}

\rubrique{Pour l'administration du sacrement, on prépare dans la
chambre du malade une petite table recouverte d'un linge
blanc; sur cette table, on dispose un crucifix entre deux
cierges allumés, de l'eau bénite, un rameau de buis bénit,
un plateau portant six petits flocons d'ouate pour essuyer
l'huile des onctions, de la mie de pain et de l'eau pour permettre
au prêtre de se purifier les doigts.}
\rubrique{En entrant dans la chambre du malade, le prêtre dit :}
\pax
\rubrique{Il dépose sur la table l'huile des infirmes, se revêt du
surplis et de l'étole violette, présente au malade le crucifïx
à baiser et l'asperge d'eau bénite, ainsi que la chambre et les
assistants, disant:}
\asperges
\adiutorium
\traduire{Intróeat, Dómine Jesu Christe, domum hanc sub nostræ humilitátis ingréssu, ætérna felícitas,
divína prospéritas serena lætítia, cáritas fructuósa, sánitas sempitérna: effúgiat ex hoc loco
accéssus d{\'æ}monum: adsint Angeli pacis, domúmque hanc déserat omnis malígna discórdia.
Magnífica, Dómine, super nos nomen sanctum tuum; et béne\x dic nostræ conversatióni:
sanctífica nostræ humilitátis ingréssum qui sanctus et qui pius es, et pérmanes cum Patre et Spíritu
Sancto in s{\'æ}cula sæculórum.}{Seigneur Jésus-Christ, faites
entrer dans cette
maison, sur les pas de votre
humble serviteur, l'éternelle
félicité, la divine prospérité,
la joie pure, la charité féconde,
la santé inaltérable; que
l'accès de ce lieu demeure interdit
aux démons ; que les
anges de la paix y accourent,
et que toute discorde maligne
en soit à jamais bannie.
Faites éclater, Seigneur, la
grandeur de votre saint nom,
et bé\x nissez notre ministère;
sanctifiez notre humble venue
en ce lieu, vous qui êtes
saint et bon, et demeurez
avec le Père et le Saint-Esprit,
dans les siècles des siècles.}
\Amen
\medskip
\traduire{Orémus, et deprecémur Dóminum nostrum Jesum Christum, ut benedicéndo bene\x dícat hoc
tabernáculum, et omnes habitántes in eo, et det eis Angelum bonum custódem et fáciat eos sibi
servíre ad considerándum mirabília de lege sua: avértat ab eis omnes contrárias potestátes:
erípiat eos ab omni formidíne, et ab omni perturbatióne, ac sanos in hoc tabernáculo custodíre
dignétur: Qui cum Patre et Spíritu Sancto vivit et regnat Deus in s{\'æ}cula sæculorum.}{Prions et supplions notre
Seigneur Jésus-Christ de
combler de ses béné\x dictions
cette maison et tous ceux qui
l'habitent; qu'il leur envoie
son ange pour les garder avec
soin ; qu'il les attache à son
service par la considération
des merveilles de sa loi ; qu'il
éloigne d'eux les puissances
ennemies; qu'il les délivre
de tout trouble et de toute
terreur, et qu'il daigne les
conserver sains et saufs dans
cette demeure. Lui qui vit
et règne avec le Père et le
Saint-Esprit, Dieu dans les
siècles des siècles.}
\Amen
\traduire{Orémus.}{Prions.}
\traduire{
Exáudi nos, Dómine sancte, Pater omnípotens, ætérne
Deus :  et míttere dignéris sanctum Angelum
tuum de cælis, qui custódiat, fóveat, prótegat, vísitet,
atque deféndat omnes habitántes in hoc habitáculo. Per Christum Dóminum nostrum.}
{ Exaucez notre prière,
Seigneur saint, Père tout
puissant, Dieu éternel ; et daignez envoyer du ciel votre saint
Ange pour qu'il garde, soutienne, protège, visite et défende tous ceux qui sont rassemblés en ce lieu. Nous vous
le demandons par Jésus-Christ
Notre-Seigneur.}
\Amen
\rubrique{En cas d'urgence, on peut omettre tout ou partie de ces oraisons.}

\titre{Confession des péchés}
\includescore{sources/confiteor}
\traduire{Misereátur tui omnípotens Deus, et, dimíssis
peccátis tuis, perdúcat te ad vitam ætérnam.}{Que le Dieu Tout-Puissant ait
pitié de vous, qu'il vous
pardonne vos péchés et vous
conduise à la vie éternelle.}
\Amen
\traduire{Indulgéntiam,~{\x}~absolutiónem, et remissiónem
peccatórum tu\'orum tríbuat tibi omnípotens
et miséricors Dóminus.}{Que le Seigneur Tout-Puissant et miséricordieux vous
accorde le pardon,~{\x}~l'absolution
et la rémission de vos péchés.}
\Amen

\titre{Administration du sacrement}
\rubrique{Étendant premièrement la main sur la tête du malade, le prêtre dit :}
\traduire{In nómine Pa\x tris, et Fí\x lii, et Spíritus~{\x}~Sancti, exstinguátur in te omnis virtus diáboli per impositiónem mánuum nostrárum, et per invocatiónem gloriósæ et sanctæ Dei Genetrícis Vírginis Maríæ, ejúsque íncliti Sponsi Joseph, et ómnium sanctórum Angelórum, Archangelórum, Patriarchárum, Prophetárum, Apostolórum, Mártyrum, Confessórum, Vírginum, atque ómnium simul Sanctórum. Amen.}
{Au nom du~{\x}~Père, et du~{\x}~Fils, et du Saint~{\x}~Esprit, que cesse toute emprise sur vous du démon ; par l'imposition de nos mains, et l'invocation de la glorieuse et sainte Mère de
Dieu, la Vierge Marie, de
son illustre époux, saint Joseph, de tous les saints anges,
archanges, patriarches, prophètes, apôtres, martyrs, confesseurs, vierges, et de tous
les saints ensemble. Ainsi soit-il.}
\rubrique{Le prêtre prend l'huile sainte et fait des onctions sur les
yeux fermés, les oreilles, les narines, la bouche fermée, les
mains et les pieds du malade, en adaptant à chaque onction les paroles ci-dessous, qui constituent la forme sacramentelle. Avec l'ouate préparée à cet effet il essuie aussitôt les parties ointes.}
\traduire{Per istam sanctam Unctió{\x}nem, et suam piíssimam misericórdiam, indúlgeat tibi Dóminus
quidquid per \emph{visum} \emph{(...aud\'itum, ..odor\'atum, ...gustum et locuti\'onem, ...tactum, ...gressum)} deliquísti. Amen.}{Par cette sainte On\x ction, et sa très douce miséricorde, daigne le Seigneur vous pardonner toutes les fautes que vous avez commises par la vue \emph{(...l'ouïe, ...l'odorat, ...le goût et la parole, ...le toucher, ...vos démarches)}. Ainsi soit-il.}
\rubrique{Si l'urgence ou j'état de santé du malade ne permettait
de faIre qu'une seule onction, on la ferait de préférence sur
le front, et l'on dirait:}
\traduire{Per istam sanctam Unctió{\x}nem, et suam piíssimam misericórdiam, indúlgeat tibi Dóminus
quidquid deliquísti. Amen.}
{Par cette sainte On{\x}ction, et sa très douce miséricorde, daigne le Seigneur vous pardonner toutes les fautes que vous avez commises. Ainsi soit-il.}
\rubrique{Les onctions terminées, le prêtre se purifie les doigts avec
de la mie de pain et de l'eau, que l'on aura soin de jeter au feu ou en terre après la cérémonie}

\titre{Prières finales}

\traduire{Kýrie eleíson. Christe. eleíson. Kýrie eleíson.}{Seigneur ayez pitié de nous, Jésus-Christ ayez pitié de nous, Seigneur ayez pitié de nous.}
\traduire{Pater noster {\red secreto usque ad}}{Notre Père {\red à voix basse}}
\traduire{\V Et ne nos indúcas in tentatiónem.}{\V Et ne nous laissez pas succomber à la tentation}
\traduire{\textbf{\R  Sed líbera nos a malo.}}{\R \textbf{Mais délivrez-nous du mal.}}
\traduire{\V  Salvum \emph{(-am)} fac servum tuum \emph{(ancíllam tuam)}.}{Sauvez donc votre serviteur \emph{(ou votre servante)}}
\traduire{\textbf{\R  Deus meus, sperántem in te.}}{\R \textbf{Qui espère en vous, ô mon Dieu}}
\traduire{\V  Mítte ei, Dómine, auxílium de sancto.}{\V De votre sanctuaire,
Seigneur, secourez-le \emph{(ou
la)}.}
\traduire{\textbf{\R  Et de Sion tuére eum \emph{(eam)}.}}{\R \textbf{Et de Sion protégez-le
\emph{(ou la)}.}}
\traduire{\V  Esto ei, Dómine, turris fortitúdinis.}{\V Soyez lui, Seigneur, une
forteresse.}
\traduire{\textbf{\R  A fácie inimíci.}}{\R \textbf{Contre l'ennemi.}}
\traduire{\V  Nihil profíciat inimícus in eo \emph{(ea)}.}{\V Que l'ennemi n'ait aucune
prise sur lui \emph{(ou elle)}.}
\traduire{\textbf{\R  Et fílius iniquitátis non appónat nocére ei.}}{\R \textbf{Et que le démon du mal
ne puisse lui nuire.}}
\traduire{\V  Dómine, exáudi oratiónem meam.}{\V Seigneur, soyez attentif
à ma prière.}
\traduire{\textbf{\R  Et clamor meus ad te véniat.}}{\R \textbf{Et que mon
cri parvienne jusqu'à vous.}}
\traduire{\dominus}{\leseigneur}
\traduire{Orémus.}{Prions.}
\traduire
{Dómine Deus, qui per Apóstolum tuum Jacóbum locútus es: Infirmátur quis in vobis? indúcat
presbýteros Ecclésiæ et orent super eum, ungéntes eum oleo in nómine Dómini: et orátio fídei
salvábit infírmum, et alleviábit eum Dóminus: et si in peccátis sit, remitténtur ei; cura,
qu{\'æ}sumus, Redémptor noster, grátia Sancti Spíritus languóres istius infírmi \emph{(infírmæ)}, ejúsque
sana vúlnera, et dimítte peccáta, atque dolóres cunctos mentis et córporis ab eo \emph{(ea)} expélle,
plenámque intérius et extérius sanitátem misericórditer redde, ut, ope misericórdiæ tuæ
restitútus \emph{(-a)}, ad prístina reparétur offícia: Qui cum Patre et eódem Spíritu Sancto vivis et regnas
Deus, in s{\'æ}cula sæculórum.}
{Seigneur Dieu qui avez
dit par l'apôtre saint Jacques: Quelqu'un d'entre
vous est-il malade? qu'il
fasse venir les prêtres de
l'Église; ils prieront sur lui
et l'oindront avec l'huile au
nom du Seigneur; et cette
prière faite avec foi sauvera le
malade, et le Seigneur le soulagera ;
et s'il est coupable
de péchés, il en obtiendra la
rémission : guérissez, ô notre
rédempteur, par la grâce du
Saint-Esprit, les infirmités de
ce \emph{(ou cette)} malade; guérissez
ses plaies et pardonnez-lui
ses péchés; soulagez son
âme et son corps et, dans
votre miséricorde, rendez-lui
une pleine santé spirituelle
et corporelle: afin que,
rétabli\emph{(e)}  par un
effet de votre bonté, il \emph{(elle)} puisse retourner à ses
devoirs. Vous qui vivez et régnez
avec le Père et le même
Saint-Esprit, Dieu dans les
siècles des siècles.}
\Amen
\traduire{Orémus.}{Prions.}
\traduire{Respíce, qu{\'æ}sumus, Dómine, fámulum tuum {\red N.} \emph{(fámulam tuam {\red N.})} in infirmitáte sui córporis
fatiscéntem, et ánimam réfove, quam creásti; ut, castigatiónibus emendátus\emph{(-a)}, se tua séntiat
medicína salvátum\emph{(-am)}. Per Christum Dóminum nostrum.}
{Considérez, Seigneur, votre serviteur N.\emph{ (ou
servante N. )} qui succombe
à la faiblesse de son corps,
et ranimez; cette âme que
vous avez créée; afin que,
rendu\emph{e} plus sage
par l'épreuve, il \emph{(ou elle)} recconnaisse qu'il \emph{(elle)} ne
doit son salut qu'à votre seule
grâce. Par le Christ, notre
Seigneur. 
}
\Amen
\traduire{Orémus.}{Prions.}
\traduire{Dómine sancte, Pater omnípotens, ætérne Deus, qui, benedictiónis tuæ grátiam ægris
infundéndo corpóribus, factúram tuam multíplici pietáte custódis: ad invocatiónem tui nóminis
benígnus assíste; ut fámulum tuum \emph{(fámulam tuam)} ab ægritúdine liberátum \emph{(-am)}, et sanitáte
donátum \emph{(-am)}, déxtera tua érigas, virtúte confírmes, potestáte tueáris, atque Ecclésiæ tuæ sanctæ,
cum omni desideráta prosperitáte, restítuas. Per Christum Dóminum nostrum.}
{Seigneur saint, Père tout
puissant, Dieu éternel, qui répandez dans le corps
des malades la grâce de vos
bénédictions, et entourez de
soins incessants vos créattures, rendez-vous à l'invocation que nous faisons de
votre nom; et après avoir
arraché à la maladie et rendu
à la santé votre serviteur
\emph{(ou servante)}, relevez-le  de votre main droite, affermissez-le \emph{(ou la)} par votre force, protégez-le (ou la) par
votre puissance et, après avoir
comblé tous ses désirs, rendez-le \emph{(ou la)} à votre sainte
Église. Par le Christ, notre Seigneur.}
\Amen
%%%%%%%%%%%%
%%%%%%%%%%%%

\titrec{Bénédiction apostolique
avec indulgence plénière}
\textit{\small La bénédiction apostolique est le suprême secours accordé
par l'Église à ses enfants malades. Elle comporte une
indulgence plénière qui, sans autre condition que l'état de
grâce, efface toutes les peines dues au péché et ouvre au
mourant, pleinement disposé à la recevoir, les portes du
paradis.}

\rubrique{Le prêtre s'est revêtu du surplis et de l'étole violette.
En entrant dans la chambre du malade, il dit:}
\pax
\rubrique{Il asperge alors d'eau bénite le malade, la chambre et les
assistants, disant : }
\asperges
\traduire{\V Adiut\'orium nostrum in n\'omine D\'omini.}{\V Notre secours est dans le Nom du Seigneur.}
\traduire{\R \textbf{Qui fecit cælum et terram.}}{\R \textbf{Qui a fait le ciel et la terre.}}
\traduire{{\red Antiphona.} Ne reminiscáris, Dómine, delícta fámuli tui \emph{(fámulæ tuæ)} : neque vindictam sumas de peccatis ejus.}
{{\red Antienne.} Ne vous souvenez pas, Seigneur, des fautes de votre serviteur \emph{(ou servante)}, et ne tirez pas vengeance de ses péchés.}

\traduire{Kýrie eleíson. Christe. eleíson. Kýrie, eleíson.}{Seigneur ayez pitié de nous, Jésus-Christ ayez pitié de nous, Seigneur ayez pitié de nous.}
\traduire{Pater noster {\red secreto usque ad}}{Notre Père {\red à voix basse}}
\traduire{\V Et ne nos indúcas in tentatiónem.}{\V Et ne nous laissez pas succomber à la tentation}
\traduire{\textbf{\R  Sed líbera nos a malo.}}{\R \textbf{Mais délivrez-nous du mal.}}
\traduire{\V  Salvum \emph{(-am)} fac servum tuum \emph{(ancíllam tuam)}.}{Sauvez donc votre serviteur \emph{(ou votre servante)}}
\traduire{\textbf{\R  Deus meus, sperántem in te.}}{\R \textbf{Qui espère en vous, ô mon Dieu}}
\traduire{\V  Dómine, exáudi oratiónem meam.}{\V Seigneur, soyez attentif
à ma prière.}
\traduire{\textbf{\R  Et clamor meus ad te véniat.}}{\R \textbf{Et que mon
cri parvienne jusqu'à vous.}}
\traduire{\dominus}{\leseigneur}
\traduire{Orémus.}{Prions.}
\traduire{Clementíssime Deus, Pater misericordiárum et Deus totíus consolatiónis, qui néminem vis
períre in te credéntem atque sperántem: secúndum multitúdinem miseratiónum tuárum réspice
propítius fámulum tuum {\red N.}, quem \emph{(fámulam tuam {\red N.}, quam)} tibi vera fides et spes christiána
comméndant.Vísita eum \emph{(eam)} in salutári tuo, et per Unigéniti tui passiónem et mortem, ómnium ei delictórum suórum remissiónem et véniam cleménter indúlge: ut ejus ánima in hora éxitus sui te
júdicem propitiátum invéniat et in Sánguine ejúsdem Fílii tui ab omni mácula ablúta, transíre ad
vitam mereátur perpétuam. Per eúmdem ...}
{Dieu très clément, Père
des miséricordes et
Dieu de toute consolation,
qui ne voulez voir périr
aucun de ceux qui croient et
espèrent en vous : jetez,
selon l'étendue de votre
commisération, un regard
bienveillant sur votre serviteur
{\red N.} \emph{(ou servante {\red N.})},
que vous recommandent
sa foi sincère et son espérance
chrétienne. Visitez cette
âme que vous voulez sauver
et, par les mérites de la passion
et de la mort de votre
Fils unique, accordez-lui la
rémission et le pardon de
toutes ses fautes; afin qu'à
l'heure où elle quittera la
terre, elle trouve en vous
un juge plein d'indulgence
et que, lavée de toute souillure
dans le sang de votre
Fils, elle mérite de passer de ce monde à la vie éternelle.
Par le même Christ, notre Seigneur.}
\Amen

\rubrique{On fait alors la confession des péchés; on dit \emph{Confiteor}
\emph{Misere\'atur tui} et \emph{Indulgéntiam} comme à la page \pageref{confiteor}}
\rubrique{Aussitôt, le prêtre prononce sur le malade la formule de la
bénédiction apostolique.}
\traduire{Dóminus noster Jesus Christus, Fílius Dei vivi, qui beáto Petro Apóstolo suo dedit potestátem
ligándi, atque solvéndi, per suam piíssimam misericórdiam recípiat confessiónem tuam, et restítuat
tibi stolam primam, quam in Baptísmate recepísti; et ego facultáte mihi ab Apostólica Sede tribúta,
indulgéntiam plenáriam et remissiónem ómnium peccatórum tibi concédo. In nómine Patris, et
Fílii,{\x} et Spíritus Sancti}
{Daigne notre Seigneur
Jésus-Christ, Fils du
Dieu vivant, qui a accordé
au bienheureux Pierre, son
apôtre, le pouvoir de lier et
de délier, accueillir dans sa
très douce miséricorde la
confession de vos fautes, et
vous restituer le vêtement
d'innocence que vous avez
reçu au baptême; et moi, en
vertu du pouvoir que le siège
apostolique m'a concédé, je
vous accorde indulgence plénière et rémission de tous
vos péchés. Au nom du Père,
et du Fils, {\x} et du Saint-Esprit.}
\Amen
\traduire{Per sacrosáncta humánæ reparatiónis mystéria remíttat tibi omnípotens Deus omnes præséntis et
futúræ vitæ poenas, paradísi portas apériat, et ad gáudia sempitérna perdúcat.}
{Par le mystère de notre
très sainte rédemption, daigne
le Dieu tout-puissant vous
remettre toutes les peines de
la vie présente et de la vie
future vous ouvrir les portes
du ciel, et vous introduire
dans la joie éternelle.}
\Amen
\traduire{Benedícat te omnípotens Deus, Pater, et Fílius, {\x} et Spíritus Sanctus.}
{Que le Dieu tout-puissant vous bénisse : le Père,
le Fils, {\x} et le Saint-Esprit.}
\Amen
\rubrique{Si le temps presse, on peut, omettant tout ce qui précède
et même la confession générale, dire seulement : {\black{Ego faculta
te mihi ab Apostlica Sede tributa,}} et ce qui suit.
Si l'urgence était extrême, on dirait seulement:}
\traduire{Ego, facultáte mihi ab Apostólica Sede tribúta, indulgéntiam plenáriam et remissiónem
ómnium peccatórum tibi concédo, et benedíco te. In nómine Patris, et Fílii, {\x} et Spíritus Sancti.}{En vertu du pouvoir que le siège apostolique m'a concédé, Je vous accorde indulgence plénière et rémission
de tous vos péchés, et je vous bénis. Au nom du Père, et du Fils, {\x} et du Saint-Esprit.}
\Amen

%%%%%%%%%%%%%%%%%%%%%%%%%%%%%%%%%%%%%%%%%%
\titrec{Rituel continu pour l'administration des sacrements des malades}
\rubrique{En entrant dans la chambre du malade, le prêtre dit :}
\pax
\rubrique{Puis après avoir déposé le Saint-Sacrement à un endroit que l'on aura pu préparer en y mettant une nappe blanche, un crucifix et deux cierges,  il se revêt du
surplis et de l'étole violette, présente au malade le crucifïx
à baiser et l'asperge d'eau bénite, ainsi que la chambre et les assistants.}
\rubrique{Puis il dit :}
%%%%%%%%%
\asperges
\adiutorium
\traduire{Intróeat, Dómine Jesu Christe, domum hanc sub nostræ humilitátis ingréssu, ætérna felícitas,
divína prospéritas serena lætítia, cáritas fructuósa, sánitas sempitérna: effúgiat ex hoc loco
accéssus dæmonum: adsint Angeli pacis, domúmque hanc déserat omnis malígna discórdia.
Magnífica, Dómine, super nos nomen sanctum tuum; et béne\x dic nostræ conversatióni:
sanctífica nostræ humilitátis ingréssum qui sanctus et qui pius es, et pérmanes cum Patre et Spíritu
Sancto in s{\'æ}cula sæculórum.}{Seigneur Jésus-Christ, faites
entrer dans cette
maison, sur les pas de votre
humble serviteur, l'éternelle
félicité, la divine prospérité,
la joie pure, la charité féconde,
la santé inaltérable; que
l'accès de ce lieu demeure interdit
aux démons ; que les
anges de la paix y accourent,
et que toute discorde maligne
en soit à jamais bannie.
Faites éclater, Seigneur, la
grandeur de votre saint nom,
et bé\x nissez notre ministère;
sanctifiez notre humble venue
en ce lieu, vous qui êtes
saint et bon, et demeurez
avec le Père et le Saint-Esprit,
dans les siècles des siècles.}
\Amen
\traduire{Orémus, et deprecémur Dóminum nostrum Jesum Christum, ut benedicéndo bene\x dícat hoc
tabernáculum, et omnes habitántes in eo, et det eis Angelum bonum custódem et fáciat eos sibi
servíre ad considerándum mirabília de lege sua: avértat ab eis omnes contrárias potestátes:
erípiat eos ab omni formidíne, et ab omni perturbatióne, ac sanos in hoc tabernáculo custodíre
dignétur: Qui cum Patre et Spíritu Sancto vivit et regnat Deus in s{\'æ}cula sæculorum.}{Prions et supplions notre
Seigneur Jésus-Christ de
combler de ses béné\x dictions
cette maison et tous ceux qui
l'habitent; qu'il leur envoie
son ange pour les garder avec
soin ; qu'il les attache à son
service par la considération
des merveilles de sa loi ; qu'il
éloigne d'eux les puissances
ennemies; qu'il les délivre
de tout trouble et de toute
terreur, et qu'il daigne les
conserver sains et saufs dans
cette demeure. Lui qui vit
et règne avec le Père et le
Saint-Esprit, Dieu dans les
siècles des siècles.}
\Amen
\traduire{Orémus.}{Prions.}
\traduire{
Exáudi nos, Dómine sancte, Pater omnípotens, ætérne
Deus :  et míttere dignéris sanctum Angelum
tuum de cælis, qui custódiat, fóveat, prótegat, vísitet,
atque deféndat omnes habitántes in hoc habitáculo. Per Christum Dóminum nostrum.}
{ Exaucez notre prière,
Seigneur saint, Père tout
puissant, Dieu éternel ; et daignez envoyer du ciel votre saint
Ange pour qu'il garde, soutienne, protège, visite et défende tous ceux qui sont rassemblés en ce lieu. Nous vous
le demandons par Jésus-Christ
Notre-Seigneur.}
\Amen
\rubrique{En cas d'urgence, on peut omettre tout ou partie de ces oraisons.}
\titre{Confession des péchés}
\rubrique{Alors le malade fait habituellement sa confession générale ; on récite alors, si ce n'est pas déjà fait, le \emph{Confiteor}}
\traduire{\textbf{Conf\'iteor Deo omnipoténti, beátæ Maríæ semper Vírgini, beáto Michaéli Archángelo, beáto Joánni Baptístæ, sanctis Apóstolis Petro et Paulo, ómnibus Sanctis et tibi Pater quia peccávi nimis cogitatióne, verbo et ópere: \emph{ (On se frappe trois fois la poitrine)} mea culpa, mea culpa, mea máxima culpa. Ideo precor beátam Maríam semper Vírginem, beátum Michaélem Archángelum, beátum Joánnem Baptístam, sanctos Apóstolos Petrum et Paulum, omnes Sanctos et te Pater, oráre pro me ad Dóminum Deum nostrum.}}
{\textbf{Je confesse à Dieu Tout-Puissant, à la bienheureuse Marie toujours Vierge, à saint Michel Archange, à saint Jean-Baptiste, aux saints Apôtres Pierre et Paul, à tous les Saints, et à vous, mon Père que j'ai beaucoup péché, par pensées, par paroles et par actions. \emph{ (On se frappe trois fois la poitrine)} C'est ma faute, c'est ma faute, c'est ma très grande faute. C'est pourquoi je supplie la bienheureuse Marie toujours Vierge, saint Michel Archange, saint Jean-Baptiste, les saints Apôtres Pierre et Paul, tous les Saints, et vous, mon Père, de prier pour moi le Seigneur notre Dieu.}}
\traduire{Misereátur tui omnípotens Deus, et, dimíssis
peccátis tuis, perdúcat te ad vitam ætérnam.}{Que le Dieu Tout-Puissant ait pitié de vous, qu'il vous pardonne vos péchés et vous conduise à la vie éternelle.}
\Amen
\traduire{Indulgéntiam,~{\x}~absolutiónem, et remissiónem
peccatórum tu\'orum tríbuat tibi omnípotens
et miséricors Dóminus.}{Que le Seigneur Tout-Puissant et miséricordieux vous
accorde le pardon,~{\x}~l'absolution
et la rémission de vos péchés.}
\Amen

\titre{Onction des malades}
\rubrique{Étendant premièrement la main sur la tête du malade, le prêtre dit :}
\traduire{In nómine Pa\x tris, et Fí\x lii, et Spíritus~{\x}~Sancti, exstinguátur in te omnis virtus diáboli per impositiónem mánuum nostrárum, et per invocatiónem gloriósæ et sanctæ Dei Genetrícis Vírginis Maríæ, ejúsque íncliti Sponsi Joseph, et ómnium sanctórum Angelórum, Archangelórum, Patriarchárum, Prophetárum, Apostolórum, Mártyrum, Confessórum, Vírginum, atque ómnium simul Sanctórum. Amen.}
{Au nom du~{\x}~Père, et du~{\x}~Fils, et du Saint~{\x}~Esprit, que cesse toute emprise sur vous du démon ; par l'imposition de nos mains, et l'invocation de la glorieuse et sainte Mère de Dieu, la Vierge Marie, de son illustre époux, saint Joseph, de tous les saints anges, archanges, patriarches, prophètes, apôtres, martyrs, confesseurs, vierges, et de tous les saints ensemble. Ainsi soit-il.}
\rubrique{Le prêtre prend l'huile sainte et fait des onctions sur les
yeux fermés, les oreilles, les narines, la bouche fermée, les
mains et les pieds du malade, en adaptant à chaque onction les paroles ci-dessous, qui constituent la forme sacramentelle. Avec l'ouate préparée à cet effet il essuie aussitôt les parties ointes.}
\traduire{Per istam sanctam Unctió\ \x nem, et suam piíssimam misericórdiam, indúlgeat tibi Dóminus
quidquid per visum \emph{(...aud\'itum, ..odor\'atum, ...gustum et locuti\'onem, ...tactum, ...gressum)} deliquísti. Amen.}{Par cette sainte On\x ction, et sa très douce miséricorde, daigne le Seigneur vous pardonner toutes les fautes que vous avez commises par la vue \emph{(...l'ouïe, ...l'odorat, ...le goût et la parole, ...le toucher, ...vos démarches)}. Ainsi soit-il.}
\rubrique{Si l'urgence ou l'état de santé du malade ne permettait
de faire qu'une seule onction, on la ferait de préférence sur
le front, et l'on dirait:}
\traduire{Per istam sanctam Unctió \x nem, et suam piíssimam misericórdiam, indúlgeat tibi Dóminus
quidquid deliquísti. Amen.}
{Par cette sainte On\x ction, et sa très douce miséricorde, daigne le Seigneur vous pardonner toutes les fautes que vous avez commises. Ainsi soit-il.}
\rubrique{Les onctions terminées, le prêtre se purifie les doigts avec
de la mie de pain et de l'eau, que l'on aura soin de jeter au feu ou en terre après la cérémonie}

\traduire{Kýrie eleíson. Christe. eleíson. Kýrie, eleíson.}{Seigneur ayez pitié de nous, Jésus-Christ ayez pitié de nous, Seigneur ayez pitié de nous.}
\traduire{Pater noster {\red secreto usque ad}}{Notre Père {\red à voix basse}}
\traduire{\V Et ne nos indúcas in tentatiónem.}{\V Et ne nous laissez pas succomber à la tentation}
\traduire{\textbf{\R  Sed líbera nos a malo.}}{\R \textbf{Mais délivrez-nous du mal.}}
\traduire{\V  Salvum \emph{(-am)} fac servum tuum \emph{(ancíllam tuam)}.}{Sauvez donc votre serviteur \emph{(ou votre servante)}}
\traduire{\textbf{\R  Deus meus, sperántem in te.}}{\R \textbf{Qui espère en vous, ô mon Dieu}}
\traduire{\V  Mítte ei, Dómine, auxílium de sancto.}{\V De votre sanctuaire,
Seigneur, secourez-le \emph{(ou
la)}.}
\traduire{\textbf{\R  Et de Sion tuére eum \emph{(eam)}.}}{\R \textbf{Et de Sion protégez-le
\emph{(ou la)}.}}
\traduire{\V  Esto ei, Dómine, turris fortitúdinis.}{\V Soyez lui, Seigneur, une
forteresse.}
\traduire{\textbf{\R  A fácie inimíci.}}{\R \textbf{Contre l'ennemi.}}
\traduire{\V  Nihil profíciat inimícus in eo \emph{(ea)}.}{\V Que l'ennemi n'ait aucune
prise sur lui \emph{(ou elle)}.}
\traduire{\textbf{\R  Et fílius iniquitátis non appónat nocére ei.}}{\R \textbf{Et que le démon du mal
ne puisse lui nuire.}}
\traduire{\V  Dómine, exáudi oratiónem meam.}{\V Seigneur, soyez attentif
à ma prière.}
\traduire{\textbf{\R  Et clamor meus ad te véniat.}}{\R \textbf{Et que mon
cri parvienne jusqu'à vous.}}
\traduire{\dominus}{\leseigneur}
\traduire{Orémus.}{Prions.}
\traduire
{Dómine Deus, qui per Apóstolum tuum Jacóbum locútus es: Infirmátur quis in vobis? indúcat
presbýteros Ecclésiæ et orent super eum, ungéntes eum oleo in nómine Dómini: et orátio fídei
salvábit infírmum, et alleviábit eum Dóminus: et si in peccátis sit, remitténtur ei; cura,
qu{\'æ}sumus, Redémptor noster, grátia Sancti Spíritus languóres istius infírmi \emph{(infírmæ)}, ejúsque
sana vúlnera, et dimítte peccáta, atque dolóres cunctos mentis et córporis ab eo \emph{(ea)} expélle,
plenámque intérius et extérius sanitátem misericórditer redde, ut, ope misericórdiæ tuæ
restitútus \emph{(-a)}, ad prístina reparétur offícia: Qui cum Patre et eódem Spíritu Sancto vivis et regnas
Deus, in s{\'æ}cula sæculórum.}
{Seigneur Dieu qui avez
dit par l'apôtre saint Jacques: Quelqu'un d'entre
vous est-il malade? qu'il
fasse venir les prêtres de
l'Église; ils prieront sur lui
et l'oindront avec l'huile au
nom du Seigneur; et cette
prière faite avec foi sauvera le
malade, et le Seigneur le soulagera ;
et s'il est coupable
de péchés, il en obtiendra la
rémission : guérissez, ô notre
rédempteur, par la grâce du
Saint-Esprit, les infirmités de
ce \emph{(ou cette)} malade; guérissez
ses plaies et pardonnez-lui
ses péchés; soulagez son
âme et son corps et, dans
votre miséricorde, rendez-lui
une pleine santé spirituelle
et corporelle: afin que,
rétabli\emph{(e)}  par un
effet de votre bonté, il \emph{(elle)} puisse retourner à ses
devoirs. Vous qui vivez et régnez
avec le Père et le même
Saint-Esprit, Dieu dans les
siècles des siècles.}
\Amen
\traduire{Orémus.}{Prions.}
\traduire{Respíce, qu{\'æ}sumus, Dómine, fámulum tuum {\red N.} \emph{(fámulam tuam {\red N.})} in infirmitáte sui córporis
fatiscéntem, et ánimam réfove, quam creásti; ut, castigatiónibus emendátus\emph{(-a)}, se tua séntiat
medicína salvátum\emph{(-am)}. Per Christum Dóminum nostrum.}
{Considérez, Seigneur, votre serviteur N.\emph{ (ou
servante N. )} qui succombe
à la faiblesse de son corps,
et ranimez; cette âme que
vous avez créée; afin que,
rendu\emph{e} plus sage
par l'épreuve, il \emph{(ou elle)} recconnaisse qu'il \emph{(elle)} ne
doit son salut qu'à votre seule
grâce. Par le Christ, notre
Seigneur. 
}
\Amen
\traduire{Orémus.}{Prions.}
\traduire{Dómine sancte, Pater omnípotens, ætérne Deus, qui, benedictiónis tuæ grátiam ægris
infundéndo corpóribus, factúram tuam multíplici pietáte custódis: ad invocatiónem tui nóminis
benígnus assíste; ut fámulum tuum \emph{(fámulam tuam)} ab ægritúdine liberátum \emph{(-am)}, et sanitáte
donátum \emph{(-am)}, déxtera tua érigas, virtúte confírmes, potestáte tueáris, atque Ecclésiæ tuæ sanctæ,
cum omni desideráta prosperitáte, restítuas. Per Christum Dóminum nostrum.}
{Seigneur saint, Père tout
puissant, Dieu éternel, qui répandez dans le corps
des malades la grâce de vos
bénédictions, et entourez de
soins incessants vos créattures, rendez-vous à l'invocation que nous faisons de
votre nom; et après avoir
arraché à la maladie et rendu
à la santé votre serviteur
\emph{(ou servante)}, relevez-le  de votre main droite, affermissez-le \emph{(ou la)} par votre force, protégez-le (ou la) par
votre puissance et, après avoir
comblé tous ses désirs, rendez-le (ou la) à votre sainte
Église. Par le Christ, notre Seigneur.}
\Amen
%%%%%%%%%%%%%%%%%
\titre{Communion}
\rubrique{Puis, tenant en main la sainte Hostie et la montrant, le prêtre dit :}
\traduire{Ecce Agnus Dei, ecce qui tollit pecc\'ata mundi.}
{Voici l'Agneau de Dieu, voici Celui qui efface les péchés du monde.}
\rubrique{On dit alors trois fois}
\traduire{\textbf{\R Dómine, non sum dignus,
ut intres sub tectum meum:
sed tantum dic verbo,
et sanábitur ánima mea}}{\textbf{\R Seigneur, je ne suis pas
digne que vous entriez sous
mon toit, mais dites seulement
une parole et mon âme sera
guérie.}}
\traduire{Accipe, frater (\emph{vel soror}), Vi\'aticum Corporis
D\'omini nostri Jesu Christi,
qui te custodiat ab hoste
mal\'igno, et perd\'ucat in vitam ætérnam.
Amen.}
{Recevez, mon frère \emph{(ou
ma soeur)}, le viatique du
Corps de notre Seigneur Jésus-Christ; qu'il vous défende contre l'ennemi, et
vous conduise à la vie éternelle. Ainsi soit-il.}
\traduire{\dominus}{\leseigneur}
\traduire{Orémus.}{Prions.}
\traduire{Dómine sancte, Pater omnípotens, ætérne Deus, te fidéliter deprecámur, ut accipiénti fratri nostro \emph{(soróri nostræ)} sacrosánctum Corpus Dómini nostri Jesu Christi Fílii tui, tam córpori,
quam ánimæ prosit ad remédium sempitérnum: Qui tecum vivit et regnat in unitáte Spíritus
Sancti Deus, per ómnia s\'æcula sæculórum}
{Seigneur saint, Père tout puissant,
Dieu éternel, nous vous demandons avec
confiance que le Corps sacré
de notre Seigneur JésusChrist,
votre Fils, que notre
frère \emph{(ou soeur)} vient de
recevoir lui soit, tant pour
le corps que pour l'âme, un
remède perpétuel. Lui qui vit
et règne avec vous en l'unité
du Saint-Esprit, Dieu dans
tous les siècles des siècles.}
\Amen

%%%%%%%%%%%%%%%
\titre{Bénédiction apostolique à l'article de la mort}

\traduire{Orémus.}{Prions.}
\traduire{Clementíssime Deus, Pater misericordiárum et Deus totíus consolatiónis, qui néminem vis
períre in te credéntem atque sperántem: secúndum multitúdinem miseratiónum tuárum réspice
propítius fámulum tuum {\red N.}, quem \emph{(fámulam tuam {\red N.}, quam)} tibi vera fides et spes christiána
comméndant.Vísita eum \emph{(eam)} in salutári tuo, et per Unigéniti tui passiónem et mortem, ómnium ei delictórum suórum remissiónem et véniam cleménter indúlge: ut ejus ánima in hora éxitus sui te
júdicem propitiátum invéniat et,in Sánguine ejúsdem Fílii tui ab omni mácula ablúta, transíre ad
vitam mereátur perpétuam. Per eúmdem Christum Dóminum nostrum.}
{Dieu très clément, Père
des miséricordes et
Dieu de toute consolation,
qui ne voulez voir périr
aucun de ceux qui croient et
espèrent en vous : jetez,
selon l'étendue de votre
commisération, un regard
bienveillant sur votre serviteur
{\red N.} \emph{(ou servante {\red N.})},
que vous recommandent
sa foi sincère et son espérance
chrétienne. Visitez cette
âme que vous voulez sauver
et, par les mérites de la passion
et de la mort de votre
Fils unique, accordez-lui la
rémission et le pardon de
toutes ses fautes; afin qu'à
l'heure où elle quittera la
terre, elle trouve en vous
un juge plein d'indulgence
et que, lavée de toute souillure
dans le sang de votre
Fils, elle mérite de passer de ce monde à la vie éternelle.
Par le même Christ, notre Seigneur.}
\Amen

\traduire{Dóminus noster Jesus Christus, Fílius Dei vivi, qui beáto Petro Apóstolo suo dedit potestátem
ligándi, atque solvéndi, per suam piíssimam misericórdiam recípiat confessiónem tuam, et restítuat
tibi stolam primam, quam in Baptísmate recepísti; et ego facultáte mihi ab Apostólica Sede tribúta,
indulgéntiam plenáriam et remissiónem ómnium peccatórum tibi concédo. In nómine Patris, et
Fílii,~{\x}~et Spíritus Sancti}
{Daigne notre Seigneur
Jésus-Christ, Fils du
Dieu vivant, qui a accordé
au bienheureux Pierre, son
apôtre, le pouvoir de lier et
de délier, accueillir dans sa
très douce miséricorde la
confession de vos fautes, et
vous restituer le vêtement
d'innocence que vous avez
reçu au baptême; et moi, en
vertu du pouvoir que le siège
apostolique m'a concédé, je
vous accorde indulgence plénière et rémission de tous
vos péchés. Au nom du Père,
et du Fils, {\x} et du Saint-Esprit.}
\Amen
\traduire{Per sacrosáncta humánæ reparatiónis mystéria remíttat tibi omnípotens Deus omnes præséntis et
futúræ vitæ poenas, paradísi portas apériat, et ad gáudia sempitérna perdúcat.}
{Par le mystère de notre
très sainte rédemption, daigne
le Dieu tout-puissant vous
remettre toutes les peines de
la vie présente et de la vie
future vous ouvrir les portes
du ciel, et vous introduire
dans la joie éternelle.}
\Amen

\rubrique{Si le temps presse, on peut, omettant tout ce qui précède
et même la confession générale, dire seulement : {\black{Ego faculta
te mihi ab Apostlica Sede tributa,}} et ce qui suit.
Si l'urgence était extrême, on dirait seulement:}
\traduire{Ego, facultáte mihi ab Apostólica Sede tribúta, indulgéntiam plenáriam et remissiónem
ómnium peccatórum tibi concédo, et benedíco te. In nómine Patris, et Fílii, {\x} et Spíritus Sancti.}
{En vertu du pouvoir que le siège apostolique m'a concédé, Je vous accorde indulgence plénière et rémission de tous vos péchés, et je vous bénis. Au nom du Père, et du Fils, {\x} et du Saint-Esprit.}
\Amen


%%%%%%%%%%%%%%%%%%%%%%%%%%%%%%%%%%%%%%%%%%%%%%%%%%%%%%%%
%%%%%%%%%%%%%%%%%%%%%%%%%%%%%%%%%%%%%%%%%%%%%%%%%%%%%%%%
%%%%%%%%%%%%%%%%%%%%%%%%%%%%%%%%%%%%%%%%%%%%%%%%%%%%%%%%
\chap
\titreb{BÉNÉDICTIONS}
\bigskip

\titrec{Bénédiction des maisons}
\titred{Bénédiction d'une Maison déjà habitée}

\textit{\small  Anciennement, le Samedi-Saint, on bénissait solennellement
toutes les maisons des fidèles. L'eau, bénite solennellement à
l'office du matin, était transportée dans toutes les demeures qui étaient ainsi comme purifiées et sanctifiées pour la célébration
des Fêtes pascales.
Outre cette bénédiction collective, l'Eglise permet les
bénédictions privées en tout temps et plusieurs fois, sur le désir
des habitants.}

\rubrique{En entrant dans la maison, le prêtre adresse le salut chrétien, en disant :}
\pax
\rubrique{Il asperge ensuite d'eau bénite les différents lieux, en disant:}
\asperges
\titred{Bénédiction d'une maison neuve}
\adiutorium
\traduire{Te Deum Patrem Omnipot\'entem
suppl\'iciter
exor\'amus pro hac domo, et
habit\'antibus ejus, ac rebus :
ut eam bene\x d\'icere, et
sancti\x fic\'are, ac bonis \'omnnibus
ampli\'are dign\'eris : tr\'ibue eis D\'omine de rore c\ae li abund\'antiam,
et de
pingu\'edine terræ vitæ subst\'antiam,
et desid\'eria voti
eorum ad aff\'ectum tuæ miserati\'onis
perd\'ucas. Ad introïtum
ergo nostrum bene\x d\'icere,
et sancti\x fic\'are dign\'eris
hanc domum sicut
bened\'icere dign\'atus es domum
Abraham, Isaac, et
Jacob : et intra par\'ietes
domus ist\'ius, Angeli tuæ
lucis inh\'abitent; e\'amque
et ejus habitat\'ores cust\'odiant.
Per Christum Dóminum nostrum.}{
Nous vous invoquons humblement,
Père tout-puissant,
pour cette demeure et
pour ses habitants avec ce
qu'elle renferme ; daignez la
bé{\x}nir, la san{\x}ctifier et la combler ;
par là de toutes sortes de biens.
Donnez, Seigneur, à ceux qui
vont l'habiter, l'abondance de
votre rosée céleste, les richesses
du sol nécessaires à la vie;
accordez-leur par votre infinie
miséricorde la réalisation de
tous leurs désirs. Oui, à notre
entrée, daignez bé{\x}nir et san{\x}ctifier
cette maison, comme vous
l'avez fait pour celle d'Abraham,
d'Isaac et de Jacob; que les
anges du ciel y établissent leur
séjour pour la garder et la protéger,
elle et tous ses habitants.}
\Amen
\rubrique{Le prêtre asperge ensuite les différentes parties de la maison.}

\titrec{Bénédiction d'un véhicule}
\adiutorium
\traduire{Propitiare, D\'omine
Deus, supplicati\'onibus
nostris, et béne\x dic
currum istum déxtera
tua sancta : adj\'unge ad
ipsum sanctos Angelos
tuos, ut omnes, qui in
eo vehéntur, l\'iberent et
cust\'odiant semper a per\'iculis
univérsis : et quem\'admodum
viro {\AE}th{\'i}opi
super currum suum sedénti,
et sacra el\'oquia legénti,
per Lev{\'i}tam tuum
Phil\'ippum fidem et gr\'atiam
contul\'isti; ita f\'amulis
tuis viam sal\'utis osténde,
qui tua gr\'atia adjuti
bon\'isque opéribus jugiter
inténti, post omnes via:
et vita: hujus variet\'ates,
ætérna g\'audia c\'onsequi
mere\'antur. Per Christum
Dominum nostrum.}{
Accueillez, Seigneur Dieu, nos supplications et bé\x nissez de votre sainte main cette voiture; attachez-y vos saints anges, qui délivreront et préserveront toujours de tout danger ceux qu'elle transportera : et de même que vous avez, par votre serviteur Philippe, donné la grâce de la foi à cet Éthiopien qui assis sur son char lisait la sainte Écriture, montrez à vos fidèles la voie du salut, de telle sorte que toujours préoccupés de faire le bien, et soutenus par votre grâce, il puissent après toutes les vicissitudes de ce voyage terrestre parvenir à la joie éternelle. Par le Christ, notre Seigneur.}
\Amen
\rubrique{Et le prêtre asperge la voiture d'eau bénite.}
%%%%%%%%%%%%
\titrec{Bénédiction du chapelet}
\adiutorium
\traduire{Omnípotens et miséricors Deus, qui propter exímiam caritátem tuam, qua dilexísti nos, Fílium tuum unigénitum, Dóminum nostrum Jesum Christum,  de cælis in terram descéndere, et de beatíssimæ Vírginis Maríæ Dóminæ nostræ utero sacratíssimo, Angelo nunciánte, carnem suscípere, crucémque ac mortem subíre, et tértia die glorióse a mórtuis resúrgere voluísti, ut nos eríperes de potestáte diáboli: 
obsecrámus imménsam cleméntiam tuam; ut hæc signa Rosárii, in honórem et laudem ejúsdem Genetrícis Fílii tui ab Ecclésia tua fidéli dicáta,  bene\x  dícas, et sanctí\x  fices, eísque tantam infúndas virtútem Spíritus\x  Sancti, ut, a quicúmque horum quódlibet secum portáverit, atque in domo sua reverénter tenúerit, et in eis ad te, secúndum hujus sanctæ Societátis institúta, divína contemplándo mystéria devóte oráverit, salúbri et perseveránti devotióne abúndet, sitque consors et párticeps ómnium gratiárium, privilegiórum, et indulgentiárum, quæ eídem Societáti per Sanctam Sedem Apostólicam concéssa fúerunt, ab omni hoste visíbili et invisíbili semper et ubíque in hoc s{\'æ}culo liberétur, et in éxitu suo ab ipsa beatíssima Vírgine María Dei Genitríci tibi plenus bonis opéribus præsentári mereátur. Per eúmdem Dóminum...in unitáte ejúsdem Spíritus.}
{Dieu tout-puissant et miséricordieux, qui avez voulu, à cause de votre charité sans limite dont vous nous avez aimé, que votre Fils unique Notre-Seigneur Jésus-Christ descende du ciel sur la terre, et naisse du sein très sacré de la Bienheureuse Vierge Marie, comme l'Ange l'a annoncé, se fasse chair, subisse la croix et la mort, et ressuscite glorieusement le troisième jour, pour nous délivrer du pouvoir du démon : nous prions votre grande clémence, pour que vous daigniez bénir {\x} ce saint Rosaire que le fidèle de votre \'Eglise a dédié à l'honneur et à la gloire de la Mère de votre Fils, ainsi que de le sanctifier {\x} et d'y insuffler tellement la vertu de l'Esprit{\x}Saint, que quiconque le porte avec lui, ou le place dans sa maison pour le vénérer, et par lui vous prie avec dévotion en contemplant les mystères divins, comme institué par la Confrérie, reçoive avec abondance une dévotion sainte et persévérante, et qu'il jouisse de toutes les grâces, privilèges et indulgences qui ont été concédés à cette même confrérie par le Siège Apostolique ; qu'il soit délivré de tous ses ennemis visibles et invisibles de ce siècle en tout temps et en tout lieu, et qu'il mérite, à la fin de cette vie, d'être présenté par cette même Vierge Marie votre Mère. Par le même Jésus-Christ votre Fils...}
\Amen

\rubrique{On peut aussi utiliser cette formule : }
\traduire{Ad laudem et glóriam Deiparæ Vírginis Maríæ, in memóriam mysteriórum vitæ, mortis et resurrectiónis ejúsdem Dómini nostri Jesu Christi, bene\x dicátur et sancti\x ficétur hæc sacratíssimi Rosárii coróna: in nómine Patris, et Fílii,~{\x}~ et Spíritus Sancti}
{À la louange et à la gloire de la Mère de Dieu, la Vierge Marie, en souvenir des Mystères de la vie, de la mort et de la résurrection de Jésus-Christ, Notre Seigneur, que soit {\x} bénit et {\x} sanctifié ce chapelet du Très Saint Rosaire au Nom du Père et du Fils {\x} et du Saint-Esprit.}
\Amen

\titrec{Bénédiction d'une image pieuse ou d'une statue}
\adiutorium
\traduire{Omnípotens sempitérne Deus, qui Sanctórum tuórum imágines \emph{(sive effigies)} sculpi aut pingi non réprobas, ut quóties illas óculis córporis intuémur tóties eórum actus et sanctitátem  ad imitándum memóriæ óculis meditémur : hanc qu{\'æ}sumus, imáginem \emph{(seu sculptúram)} in honórem et memóriam \emph{Unigéniti Fílii tui Dómini nostri Jesus Christi (vel beatíssimæ Vírginis Maríæ, matris Dómini nostri Jesu Christi, vel beáti N., Apóstoli tui, vel Mártyris, vel Pontíficis, vel Confessóris, vel  beátæ N., Vírginis, vel Mártyris)} adaptátam,  bene\x dícere, et sancti\x  ficáre dignéris: et præsta ut quicúmque coram illa \emph{Unigénitum Fílum tuum (vel beátissimum Vírginem  vel gloriosum Apostolum, vel  Mártyrem, vel Pontíficem, vel Confessórem tuum, vel gloriósam Vírginem, vel Mártyrem)} suppliciter cólere et honoráre studúerit, illíus méritis et obténtu a te grátiam in præsenti et ætérnam glóriam obtineat in futáram. Per  \emph{(eúmdem)} Christum Dóminum nostrum.}
{Dieu Notre Père, Éternel et Tout-Puissant, Vous n'avez pas interdit de sculpter ou de peindre l'image de vos saints, pour qu'en les regardant avec les yeux de notre corps, nous puissions chaque fois nous rappeler leur vie et méditer leur sainteté. Daignez {\x} bénir et rendre {\x} sainte cette image \emph{(ou statue)} destinée à rappeler et honorer \emph{votre Fils unique, Notre Seigneur Jésus-Christ (ou la bienheureuse Vierge Marie, Mère de Notre Seigneur Jésus-Christ, ou votre bienheureux apôtre..., martyr..., évêque..., confesseur..., bienheureuse vierge...)}. Faites que tous ceux qui penseront, en la voyant, à honorer et vénérer votre Fils unique, \emph{(ou la bienheureuse Vierge, le glorieux apôtre, martyr, évêque, confesseur, la glorieuse vierge, ou martyre)} obtiennent de Vous, par ses mérites et sa protection, la grâce dans la vie présente et la gloire éternelle dans la vie future. Par \emph{(le même} Jésus-Christ, notre Seigneur et notre Dieu, qui vit et règne avec Vous en l'unité du Saint-Esprit, pour les siècles des siècles.}
\Amen


\titrec{Bénédiction de la médaille miraculeuse}
\adiutorium
\traduire{Omnípotens et miséricors Deus, qui per multíplices immaculátæ Maríæ Vírginis apparitiónes in terris mirabília júgiter pro animárum salúte operári dignátus es: super hoc numísmatis signum, tuam bene\x dictiónem benígnus infúnde; ut pie hoc recoléntes ac devóte gestántes et illíus patrocínum séntiant et tuam misericórdiam consequántur. Per Christum Dóminum nostrum.}
{Dieu Notre Père, Tout-Puissant et Miséricordieux, qui par les nombreuses apparitions sur la terre de l'Immaculée Vierge Marie, avez voulu {\oe}uvrer plus efficacement au salut des âmes: nous Vous prions de répandre votre bénédiction \x sur ces médailles ; afin que ceux qui les porteront avec piété et dévotion en éprouvent la protection et jouissent de votre Miséricorde. Par Notre Seigneur Jésus-Christ votre Fils qui vit et règne avec vous en l'unité du Saint-Esprit pour les siècles des siècles.}
\Amen

\rubrique{Le prêtre asperge alors la médaille}
\titred{Imposition de la médaille}
\rubrique{Le prêtre impose la médaille en disant :}
\traduire{Accipe sanctum Numísma, gesta fidéliter, et digna veneratióne proséquere: ut piíssima immaculáta cælórum Dómina te prótegat atque deféndat; et pietátis suæ prodígia rénovans, quæ a Deo supplíciter postuláveris, tibi misericórditer ímpetret, ut vivens ac móriens in matérno ejus ampléxu felíciter requiéscas. Amen.}{
Recevez la médaille miraculeuse, portez-la fidèlement et entourez-la du respect convenable : afin que la très sainte et Immaculée Reine des Cieux vous protège et vous défende ; et que, renouvelant pour vous les prodiges de sa bonté, elle sollicite miséricordieusement tout ce que vous demanderez à Dieu dans vos prières, et qu'ainsi vous puissiez vivre et mourir paisiblement dans ses bras maternels. Ainsi soit-il}
\rubrique{Il dit ensuite}
\traduire{Kýrie, eléison, Christe, eléison, Kýrie, eléison.}{Seigneur ayez pitié de nous, Jésus-Christ ayez pitié de nous, Seigneur ayez pitié de nous.}
\traduire{Pater noster \rubrique{secreto usque ad}}{Notre Père \rubrique{en silence jusqu'à}}
\traduire{\V Et ne nos indúcas in tentatiónem.}{\V Et ne nous laissez pas succomber à la tentation.}
\traduire{\textbf{\R Sed líbera nos a malo.}}{\textbf{\R Mais délivrez-nous du mal}}
\traduire{\V Regína sine labe origináli concépta.}{\V Reine conçue sans le péché originel}
\traduire{\R \textbf{Ora pro nobis.}}
{\textbf{\R Priez pour nous.}}

\traduire{\V D\'omine ex\'audi orati\'onem meam.}{\V Seigneur exaucez ma prière.}
\traduire{\R \textbf{Et clamor meus ad te véniat.}}{\R \textbf{Et que mon cri parvienne jusqu'à vous.}}
\traduire{\dominus}{\leseigneur}
\traduire{Orémus.}{Prions.}
\traduire{Dómine Jesu Christe, qui beatíssimam Vírginem Maríam matrem tuam ab orígine immaculátam innúmeris miráculis claréscere voluísti: concéde; ut ejúsdem patrocínium semper implorántes gáudia consequámur ætérna: Qui vivis et regnas in s\'æcula sæculórum.}{
Seigneur Jésus-Christ qui avez voulu glorifier la bienheureuse Vierge Marie votre Mère immaculée dès son origine par des miracles sans nombre : faites que tous ceux qui imploreront sa protection sans se lasser pendant leur vie, parviennent un jour au bonheur éternel : Vous qui vivez et régnez, dans les siècles des siècles.}
}
\Amen

\tableofcontents

\end{document}