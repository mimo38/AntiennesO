\documentclass[%
fontsize=12%
,a5paper%
,DIV=15%
]{scrartcl}

\usepackage{adaptateur}
\usepackage{gredocument}
\usepackage{autres}
\usepackage{dialogues}
\usepackage{mudoc}
\usepackage{pdfpages,transparent,array,ltablex}

\RequirePackage{needspace}
\RequirePackage{graphicx}
%\RequirePackage[table]{xcolor}
%\RequirePackage{keycommand,ifthen,calc}
%\RequirePackage{lettrine}
\RequirePackage[autocompile]{gregoriotex}

\title{Rituel à l'usage des fidèles}
\author{}
\date{}

\makeindex


\begin{document}
\maketitle
\titre{Objet de ce fascicule}
La Sainte Église, Notre Mère,a renfermé dans six livres principaux
l'ensemble des actes religieux qui se rapportent au
culte de Dieu et qu'on appelle pour cette raison les \emph{Livres
liturgiques}. Ce sont le Missel, le Bréviaire, le Rituel, le Pontifical,
le Cérémonial des Évêques et le Martyrologe.

Le \emph{Rituel} renferme les rites sacrés qui accompagnent l'administration
des Sacrements et des autres fonctions saintes accomplies
pour attirer sur nous les bénédictions du ciel. De là
le nom de \emph{Rituel des Fidèles}, donné à la présente publication,
destinée à instruire plus parfaitement les fidèles sur les Rites
de ces différents actes liturgiques.

\titre{Avantages}
L'Église veut que les fidèles soient instruits parfaitement
de ces rites sacrés. Le Concile de Trente, en effet, s'exprime
ainsi sur ce point  :

\og Afin,que le peuple s'approche des Sacrements avec plus
de respect et de ferveur, le saint Concile ordonne à tous les
évêques qui auraient à les administrer d'en expliquer auparavant,
et de manière à être compris, la pratique et l 'efficacité.
Ils veilleront aussi à ce que les curés, si cela se peut commodément
et s'il en est besoin, donnent les mêmes explications avec
beaucoup de sagesse et de piété et en langue vulgaire.\fg \footnote{Concile de Trente, Sess. 24, De Rej., chap. vii}

Pour faciliter cet enseignement, rendu souvent difficile
aujourd'hui par les nécessités plus urgentes d'une prédication
tout à fait élémentaire, nous avons voulu vulgariser la liturgie
des Sacrements. Ces explications sur l'efficacité, la pratique,
l 'utilité et les rites des Sacrements ne peuvent qu'instruire,
édifier et bien préparer les fidèles à les recevoir. Il en est tant
qui, plus ou moins instruits de la doctrine sacramentelle, ne
comprennent rien aux cérémonies dont ils sont les témoins.
\og Ces rites, cependant, dit le Catéchisme du Concile de Trente\footnote{2e partie, n$^o$ 16}, expriment les effets des Sacrements et les
rendent comme sensibles aux yeux des fidèles qui en comprennent mieux la sainteté. Leur foi, leur charité, leurs sentiments
surnaturels en sont excités davantage; aussi faut-il
avoir soin que ces cérémonies si touchantes et si instructives
ne leur soient pas inconnues.\fg


\versio{Indulgéntiam,~{\x}~absolutiónem, et remissiónem
peccatórum tu\'orum tríbuat tibi omnípotens\footnote{C'est une note pour essai}
et miséricors Dóminus.}{Que le Seigneur Tout-Puissant et miséricordieux vous
accorde le pardon,~{\x}~l'absolution
et la rémission de vos péchés.}
\Amen
\rubrique{Puis, tenant en main la sainte Hostie et la montrant, le prêtre dit :}
\traduire{Ecce Agnus Dei, ecce qui tollit pecc\'ata mundi}{Voici l'Agneau de Dieu, voici Celui qui efface les péchés du monde.}
\rubrique{On dit alors trois fois}
\traduire{\textbf{\R Dómine, non sum dignus,
ut intres sub tectum meum:
sed tantum dic verbo,
et sanábitur ánima mea}}{\textbf{\R Seigneur, je ne suis pas
digne que vous entriez sous
mon toit, mais dites seulement
une parole et mon âme sera
guérie.}}

\index{un}
\index{deux}
\index{trois}
\index{quatre}
\titreb{Test de gros titre}
\titrec{subsection}
\titred{sousous}
\paragraph{para}

Voici l'index :
\printindex

\newpage
\tableofcontents

\end{document}