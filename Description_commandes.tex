\documentclass[%
fontsize=10%
,a5paper%
,DIV=15%
]{scrartcl}
%scrartcl

\usepackage{gredocument}
\usepackage{psaume}

        \definecolor{rubrum}{rgb}{.6,0,0}
        \def\rubrum{\color{black}}%%%%%%%mettre"\def\rubrum{\color{rubrum}}" pour avoir le texte adéquat en rouge
        \def\nigra{\color{black}}
        %    \redlines
        %    \definecolor{gregoriocolor}{rgb}{.6,0,0}
        %
        %\let\red\rubrum

\title{\centrer{Voici le titre}}%Définit le titre de document
\author{Et l'Auteur}
\date{et la date}

\makeindex %Générer un index

\begin{document}%debut document%%%%%%%%%%%%%%%%%%%%%%%%%%%%%%%%%%%%%%%%%%%%%%%%%%
\maketitle% afficher le titre,auteur et date
\newpage

\redlines % lignes rouges
            %\gresetlinecolor{153}{0}{0}
            \definecolor{gregoriocolor}{rgb}{% définit la couleur précise des lignes des partitions
                .6,0,0%
            }

\begin{center}
\greseparator{2}{10} %insère une ligne {le style (de 1 à 5)}{taille}
\end{center}

\titre[espace=8,table=idem,matiere=non]{Et le deuxième titre} %insère un titre [espace=espace-avant-titre,table=ou idem ou non ou (titre a afficher dans l'index),matiere=ou idem ou non ou (titre a afficher dans la table des matières) 
\cantus{Alleluia}{TotaPulchraEs}{3}{4}

Plus on étudie le langage\Latex  ou \Latexe, plus on est enthousiaste
\end{document}